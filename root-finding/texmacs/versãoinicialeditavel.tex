\documentclass{article}
\usepackage[portuguese]{babel}
\usepackage{amsmath,graphicx}

%%%%%%%%%% Start TeXmacs macros
\catcode`\>=\active \def>{
\fontencoding{T1}\selectfont\symbol{62}\fontencoding{\encodingdefault}}
\newcommand{\tmop}[1]{\ensuremath{\operatorname{#1}}}
%%%%%%%%%% End TeXmacs macros

\begin{document}

Comecemos pela equa{\c c}{\~a}o da densidade. Para chegarmos na massa,
queremos resolver a integral $\bigiiint_V \rho \tmop{dV}$. Por{\'e}m, {\rho}
n{\~a}o {\'e} vari{\'a}vel, podendo, portanto, ser tirado da integral, $\rho
\bigiiint_V 1 \tmop{dV}$. Essa integral, por{\'e}m, se reduz ao volume, temos
ent{\~a}o {\rho}V. Nos bastar{\'a}, para conferirmos a f{\'o}rmula $m =
\frac{4 \pi r \tmr{^3} }{3} \rho$ que $V = \frac{4 \pi r \tmr{^3}}{3}$. Essa
f{\'o}rmula para o volume poder{\'a} ser testada juntamente com a f{\'o}rmula
para o volume da calota esf{\'e}rica, j{\'a} que a esfera ser{\'a} uma calota
esf{\'e}rica com $h = 2 R$.

Tomemos a esfera:

\raisebox{0.0\height}{\includegraphics[width=14.8741637150728cm,height=12.4056801784074cm]{vers{\~a}oinicialeditavel-1.pdf}}

Para a f{\'o}rmula da calota esf{\'e}rica, consideremos um elemento de
{\'a}rea da nossa esfera, que ser{\'a}

um cilindro com {\'a}rea da base equivalente ao c{\'i}rculo da esfera quando
cortada pelo plano z = h e altura dH.

Teremos o seguinte, num corte pelo plano xz ou yz:

\raisebox{0.0\height}{\includegraphics[width=6.72625606716516cm,height=5.92468516332153cm]{vers{\~a}oinicialeditavel-2.pdf}}

Isso significa que o c{\'i}rculo que ser{\'a} base do nosse elemento de
{\'a}rea ter{\'a}, por Pit{\'a}goras, raio:
\begin{eqnarray*}
  \tmr{R^2} = (R - h)^2 + r^2 &  & \\
  r = \sqrt{R^2 - (R - h)^2 } &  & \\
  r = \sqrt{R^2 - R^2 + 2 \tmop{Rh} - h^2} &  & \\
  r = \sqrt{2 \tmop{Rh} - h^2 } &  & 
\end{eqnarray*}
Para o caso de h>R, teremos um tri{\^a}ngulo similar, mas com cateto h-R ao
inv{\'e}s de R-h. Para

a f{\'o}rmula, que usa apenas o quadrado dessa diferen{\c c}a, isso n{\~a}o
ir{\'a} importar. Nosso elemento de

volume, portanto, ser{\'a} igual a
\begin{eqnarray*}
  \tmop{dV} = \pi r^2 \tmop{dh} &  & \\
  \tmop{dV} = \pi \left( \sqrt{2 \tmop{Rh} - h^2 } \right)^2 \tmop{dh} &  & \\
  \tmop{dV} = \pi (2 \tmop{Rh} - h^2) \tmop{dh} &  & 
\end{eqnarray*}
Para termos o volume, teremos que integrar esse valor do zero at{\'e} a altura
da calota, que ser{\'a}
\[ \int_0^h \pi (2 \tmop{Rh}' - h^{\prime 2}) \tmop{dh}' \]
Essa integral definida ter{\'a} como solu{\c c}{\~a}o
\[ V = \pi \left( \tmop{Rh}^2 - \frac{h^3}{3} \right) \]
Que significa que a f{\'o}rmula para o volume da calota esf{\'e}rica do
enunciado est{\'a} correta.

Para h = 2R, teremos
\begin{eqnarray*}
  \pi \left( R (2 R)^2 - \frac{(2 R)^3 }{3} \right) &  & \\
  \pi \left( 4 R^3 - \frac{8 R^3 }{3} \right) &  & \\
  \frac{4 \pi R^3}{3} &  & 
\end{eqnarray*}
O que confirma a f{\'o}rmula para o volume da esfera que busc{\'a}vamos
testar.

Com as f{\'o}rmulas em m{\~a}os, passemos para o problema. Queremos o ponto de
equil{\'i}brio da esfera.

Temos da segunda lei de Newton que esse ser{\'a} o ponto em que as for{\c c}as
resultantes se somam.

Existem, atuando na esfera duas for{\c c}as, o peso, cujo m{\'o}dulo {\'e}
dado pela multiplica{\c c}{\~a}o da massa

por uma constante g e cuja dire{\c c}{\~a}o ser{\'a} para baixo, e o empuxo,
que conforme o princ{\'i}pio de

Arquimedes ter{\'a} m{\'o}dulo igual ao peso do l{\'i}quido deslocado e
dire{\c c}{\~a}o para cima. Como o volume

de l{\'i}quido deslocado ser{\'a} igual ao volume submerso da calota, teremos
ent{\~a}o o seguinte
\begin{eqnarray*}
  \tmop{mg} - V_{\tmop{desloc}} d_{{\'a} \tmop{gua}} g = 0 &  & \\
  \frac{4 \pi r^3 \tmop{dg}}{3} - \pi \left( \tmop{Rh}^2 - \frac{h^3}{3}
  \right) d_{{\'a} \tmop{gua}} g = 0 &  & \\
  \frac{4 r^3 d}{3} - \left( \tmop{Rh}^2 - \frac{h^3}{3} \right) d_{{\'a}
  \tmop{gua}} = 0 &  & 
\end{eqnarray*}
Utilizando para a {\'a}gua a densidade de 1 (assumindo aqui unidades
convencionais do SI) e para a

esfera o valor dado no enunciado, 0,6, teremos
\begin{eqnarray*}
  \frac{2, 4 r^3}{3} - \left( \tmop{Rh}^2 - \frac{h^3}{3} \right) = 0 &  & \\
  2, 4 r^3 - (3 R^{} h^2 - h^3) = 0 &  & \\
  h^3 - 3 \tmop{Rh}^2 + 2, 4 r^3 = 0 &  & 
\end{eqnarray*}
Agora, queremos encontrar h em fra{\c c}{\~a}o de R. Convertemos a nossa
equa{\c c}{\~a}o para um sistema de

unidades em que, para comprimentos, R = 1.
\[ h^3 - 3 h^2 + 2, 4 = 0 \]
Temos ent{\~a}o a fun{\c c}{\~a}o para a qual queremos buscar uma raiz.
Lembremos que s{\'o} ter{\~a}o significado

f{\'i}sico as solu{\c c}{\~o}es em que $0 \leq h \leq 2$, portanto apenas
essas nos interessar{\~a}o.

A derivada dessa fun{\c c}{\~a}o ser{\'a}:
\[ 3 h^2 - 6 h \]
Que tem raiz em 0 e 2. Por conta da derivada ser uma fun{\c c}{\~a}o
cont{\'i}nua, isso significa que, nesse

intervalo, a fun{\c c}{\~a}o ou {\'e} estritamente crescente ou estritamente
decrescente, n{\~a}o podendo, portanto,

passar duas vezes pelo zero. J{\'a} que $0^3 - 3 \times 0^2 + 2, 4 = 2, 4 > 0$
e $8 - 3 \times 4 + 2, 4 = - 1, 6 < 0$ teremos uma

raiz nesse intervalo, que ser{\'a} por isso {\'u}nica.

\

Dado que a fun{\c c}{\~a}o {\'e} um polin{\^o}mio sua resposta se encontra
num intervalo bem definido e sabemos

ser a {\'u}nica raiz nesse intervalo, al{\'e}m de sua derivada ser de
c{\'a}lculo simples, o problema maior

parece ser n{\~a}o a converg{\^e}ncia do algoritmo, mas sim a velocidade da
converg{\^e}ncia. Para maximizar

essa velocidade, dentre os m{\'e}todos que estudamos, o m{\'e}todo de Newton
parece ser o ideal.

O c{\'o}digo para a aplica{\c c}{\~a}o desse m{\'e}todo ser{\'a} fornecido
junto com o PDF, mas teremos como

resultado final para o m{\'e}todo de Newton, com suposi{\c c}{\~a}o inicial de
1 e toler{\^a}ncia de $10^{- 12}$, o valor de h = 1,134137845704535, ap{\'o}s
dezesseis itera{\c c}{\~o}es do m{\'e}todo.

Isso significa que h ser{\'a}, aproximadamente, 1,134137845704535R.

Esse valor pode ser testado caso o lancemos na fun{\c c}{\~a}oo original, onde
teremos a resposta de que os valores do empuxo e da gravidade, ainda que
n{\~a}o exatamente iguais, ir{\~a}o diferir por, aproximadamente, $4, 44089
\times 10^{- 15} R^3$. Podemos notar algumas coisas. Em primeiro lugar, o erro
da nossa solu{\c c}{\~a}o ser{\'a}, em fra{\c c}{\~a}o da massa, constante,
j{\'a} que ambos tem a mesma depend{\^e}ncia c{\'u}bica de R. O erro absoluta
da for{\c c}a, por{\'e}m ir{\'a} crescer em propor{\c c}{\~a}o de $R^3$,
significando que, mesmo que ela n{\~a}o seja capaz de produzir acelera{\c
c}{\~a}o apreci{\'a}vel na esfera, teremos a presen{\c c}a de uma for{\c c}a
resultante de m{\'o}dulo consider{\'a}vel no sistema para valores maiores de
R, o que pode tornar a solu{\c c}{\~a}o inaplic{\'a}vel para esses valores,
caso se busque uma resultante zero, ao inv{\'e}s de somente uma esfera sem
movimento.

\

\

\end{document}
